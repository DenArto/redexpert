\documentclass{article}
\usepackage[utf8]{inputenc}
\usepackage[T2A,T1]{fontenc}
\usepackage[english,russian]{babel}
\usepackage{amsmath}
\usepackage{amssymb,amsfonts,textcomp}
\usepackage{fancyhdr}
\usepackage{array}
\usepackage{longtable}
\usepackage{hhline}
\usepackage[pdftex]{graphicx}
\usepackage[colorlinks=true,linkcolor=blue,unicode, pdftex]{hyperref}
\makeatletter
\newcommand\arraybslash{\let\\\@arraycr}
\makeatother
\setlength\tabcolsep{1mm}
\renewcommand\arraystretch{1.3}
\newcounter{Table}[section]
\renewcommand\theTable{\thesection.\arabic{Table}}
\parindent=1cm
\usepackage[letterpaper, left=3cm, right=2.5cm, top=2.5cm, bottom=2.5cm]{geometry}
\usepackage{color}
\usepackage{framed}
\usepackage{titlesec}
\usepackage{fancybox}
\usepackage[tikz]{bclogo}
\usepackage{xhfill}
\usepackage{booktabs}
\usepackage{listingsutf8}
\usepackage{float}
\usepackage{tablefootnote}
\usepackage{colortbl} 
\usepackage{type1ec} % для четкости шрифта
\usepackage[numbered]{bookmark} % для корректного отображения оглавления в ридере
\usepackage{arydshln} %для пунктирных линий
\usepackage[framemethod=tikz]{mdframed}
\usepackage{multirow}


\mdfsetup{%
	middlelinewidth=0.5pt,
	backgroundcolor=white!98!black,	
	roundcorner=5pt,
	innerleftmargin = 10pt,
	innertopmargin=5pt,
	skipabove = 10pt,
	skipbelow = 5pt,
	shadow=true,
	shadowsize=4pt,
	nobreak=false,
	rightmargin=0pt,
	align=right, 
}
\global\mdfdefinestyle{importantstyle}{%
	linecolor=red,linewidth=4pt,%
	topline=false,bottomline=false, rightline = false, %
	backgroundcolor=red!8,%
	shadow=false
}

\global\mdfdefinestyle{examplestyle}{%
	middlelinewidth=0.5pt,
	backgroundcolor=white!98!black,	
	roundcorner=5pt,
	innerleftmargin = 10pt,
	innertopmargin=5pt,
	skipabove = 10pt,
	skipbelow = 5pt,
	shadow=true,
	shadowsize=4pt,
	nobreak=false,
	rightmargin=0pt,
	align=right, 
}


\global\mdfdefinestyle{remarkstyle}{%
	linecolor=green!50!black,linewidth=4pt,%
	topline=false,bottomline=false, rightline = false, %
	backgroundcolor=green!8,%
	shadow=false
}

\mdfdefinestyle{theoremstyle}{%
	frametitlebackgroundcolor=white!40!black,
	frametitlealignment = \raggedright,
	frametitlerule=true,
	frametitlefont={\normalfont\bfseries\color{white}},
	frametitlebelowskip = 4pt,
	frametitleaboveskip = 4pt,
}


\newcounter{redlisting}[section]
\renewcommand{\theredlisting}{\thesection.\arabic{redlisting}}
\newcommand{\li}{\refstepcounter{redlisting} Листинг \theredlisting. }
\newcommand{\enli}{\refstepcounter{redlisting} Listing \theredlisting. }


\newenvironment{redlisting}[1]{\begin{mdframed}[style=theoremstyle,frametitle={\li #1}]\flushleft}{\end{mdframed}}

\newenvironment{redexample}{\begin{mdframed}\flushleft}{\end{mdframed}}

\newenvironment{redimportant}{\begin{mdframed}[style=importantstyle]}{\end{mdframed}}

\newenvironment{redremark}{\begin{mdframed}[style=remarkstyle]}{\end{mdframed}}


\usepackage{comment}

%\usepackage{parskip}
\usepackage{enumitem}

\usepackage{indentfirst}  % абзацный отступ после заголовка
\graphicspath{{SQLGuide-img/}} %

% другой счетчик для приложений (по \ref выдает A,B,C...)
\newcounter{application}
\renewcommand{\theapplication}{\Asbuk{application}}
\newcommand{\appcount}{\refstepcounter{application}}

\newcounter{test}
\renewcommand{\thetest}{\Asbuk{test}}
\newcommand{\apptest}{\refstepcounter{test}}

% чтобы в оглавлении было не присто А Б, а Приложение А, Приложение Б
\renewcommand\appendixname{Приложение}

\makeatletter
\def\redeflsection{\def\l@section{\@dottedtocline{1}{0em}{8em}}}
\renewcommand\appendix{\par
	\setcounter{application}{0}%
	\setcounter{subsection}{0}%
	\def\@chapapp{\appendixname}%
	\addtocontents{toc}{\protect\redeflsection}	
	\def\thesection{\appendixname\hspace{0.2cm}\Asbuk{application}}
	}
\makeatother


% чтобы разделы в оглавлении были не жирным а обычным шрифтом
\makeatletter 
\renewcommand*\l@section[2]{% 
	\ifnum \c@tocdepth >\z@ 
	\addpenalty\@secpenalty 
	\addvspace{1.0em \@plus\p@}% 
	\setlength\@tempdima{1.5em}% 
	\begingroup 
	\parindent \z@ \rightskip \@pnumwidth 
	\parfillskip -\@pnumwidth 
	\leavevmode 
	\advance\leftskip\@tempdima 
	\hskip -\leftskip 
	#1\nobreak\hfil \nobreak\hb@xt@\@pnumwidth{\hss #2}\par 
	\endgroup 
	\fi} 
\makeatother 


\emergencystretch=25pt %борьба с переполнением строк


\renewcommand{\thetable}{\thesection.\arabic{table}} % нумерация таблиц с номером section
\renewcommand{\thefigure}{\thesection.\arabic{figure}} % нумерация рисунков с номером section

\makeatletter
\@addtoreset{table}{section} % сбросит счетчик таблиц в начале каждого раздела. 
\makeatother

\makeatletter
\@addtoreset{figure}{section} % сбросит счетчик рисунков в начале каждого раздела. 
\makeatother

% пакет оформления блоков "Внимание" и "Пример"
\usepackage{tcolorbox}
\tcbuselibrary{most}

%пакет для работы с плавающими объектами
\usepackage{afterpage}

%оформление таблиц и рисунков по ГОСТ
\usepackage[tableposition=top]{caption}
\usepackage{subcaption}
\DeclareCaptionLabelFormat{gostfigure}{Рисунок #2} 
\DeclareCaptionLabelFormat{gosttable}{Таблица #2} 
\DeclareCaptionLabelSeparator{gost}{~---~}
\captionsetup{labelsep=gost}
\captionsetup[figure]{labelformat=gostfigure}
\captionsetup[table]{labelformat=gosttable,justification=raggedright,slc=off} 

\newcommand{\vspacebf}[1]{\vspace{5pt}\noindent \textbf{#1}}

%задает цвет гиперссылок 
\hypersetup{urlcolor=blue}

\tcbset{
	enhanced,
	colback=black!2,
	boxrule=0.4pt,
	fonttitle=\bfseries
}



%задает пустое пространство до и после таблицы
\setlength{\LTpre}{5pt}
\setlength{\LTpost}{5pt}

\newenvironment{redparam}
{\vspace{10pt}\bf\parindent=0cm}
{\par}

\newenvironment{redparam1}
{\vspace{5pt}\ttfamily\parindent=0cm}
{}


\newenvironment{rederror}
{\par\vspace{5pt}\ttfamily}
{\par\vspace{5pt}}

\newcommand{\sectionbreak}{\clearpage}

\titleformat{\section}
{\huge}
{\thesection}{1em}{}

\titleformat{\subsection}
{\LARGE}
{\thesubsection}{1em}{}

\titleformat{\subsubsection}
{\Large}
{\thesubsubsection}{1em}{}

\titleformat{\paragraph}
{\large}
{\theparagraph}{1em}{}

  
\title{}
\author{Red Soft Corporation}
\date{}

\setlength{\headheight}{24pt}
\pagestyle{fancy}


\newcommand{\anonsection}[1]{\phantomsection\section*{#1}\addcontentsline{toc}{section}{#1}}


\newcommand{\anonsubsection}[1]{\phantomsection\subsection*{#1}\addcontentsline{toc}{subsection}{#1}}

\newcommand{\anonsubsubsection}[2][]{\phantomsection%
	\subsubsection*{#2}%
	\ifthenelse{\equal{#1}{}}{\addcontentsline{toc}{subsubsection}{#2}}{\addcontentsline{toc}{subsubsection}{#1}}%
	}





\newcommand{\beginrudoc}[1]{
\fancyhf{}
\thispagestyle{empty}

\topskip0pt
\vspace*{\fill}

\begin{flushright}
\Huge {\xhrulefill{red}{2mm}\color{red} RedExpert} \\
\LARGE Версия 2021.07\\
#1
\end{flushright}

\vspace*{\fill}

\newpage

\fancyhead[R]{#1\\\rightmark}
\fancyfoot[C]{\xhrulefill{red}{2mm} Стр. \thepage}

\setcounter{tocdepth}{10}
\renewcommand\contentsname{Содержание}
\tableofcontents
}


%печать одинарных кавычек
\newcommand{\rr}[1]{{\textquotesingle}#1{\textquotesingle}}

%печать двойных кавычек
\newcommand{\pp}[1]{"#1{}"}

\newcommand{\ttt}{\texttt}

%для приложения 5 в SQL
\newcommand{\hl}[1]{\hyperlink{#1}{#1}}

\setlist[enumerate]{itemindent=1.3mm, topsep = 1mm, parsep=0pt, leftmargin=45pt} %чтобы списки были более сжатыми
\setlist[itemize]{itemindent=1.3mm, topsep = 1mm, parsep=0pt, leftmargin=45pt}   %чтобы списки были более сжатыми

